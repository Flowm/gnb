\chapter{Detailed Report}\label{chapter:detailed_report}

\section{Tool description}

TODO:\newline
compare the 8 different tools used by your team. Which vulnerabilities can each
tool find? Which vulnerabilities can they not find?\newline

\subsection*{Tool 1}
\begin{itemize}
	\item \textbf{Used by}\\ Name Surname
	\item \textbf{Used for}\\ Sample Text - description of the tool
	\item \textbf{Useful in}\\ \vulnref{OTG-AUTHN-004} and \vulnref{OTG-AUTHN-005}
\end{itemize}

\subsection*{Tool 2}
\begin{itemize}
	\item \textbf{Used by}\\ Name Surname
	\item \textbf{Used for}\\ Sample Text - description of the tool
	\item \textbf{Useful in}\\ \vulnref{OTG-AUTHN-004} and \vulnref{OTG-AUTHN-005}
\end{itemize}

\subsection*{Tool 3}
\begin{itemize}
	\item \textbf{Used by}\\ Name Surname
	\item \textbf{Used for}\\ Sample Text - description of the tool
	\item \textbf{Useful in}\\ \vulnref{OTG-AUTHN-004} and \vulnref{OTG-AUTHN-005}
\end{itemize}

\subsection*{Tool 4}
\begin{itemize}
	\item \textbf{Used by}\\ Name Surname
	\item \textbf{Used for}\\ Sample Text - description of the tool
	\item \textbf{Useful in}\\ \vulnref{OTG-AUTHN-004} and \vulnref{OTG-AUTHN-005}
\end{itemize}

\subsection*{Tool 5}
\begin{itemize}
	\item \textbf{Used by}\\ Name Surname
	\item \textbf{Used for}\\ Sample Text - description of the tool
	\item \textbf{Useful in}\\ \vulnref{OTG-AUTHN-004} and \vulnref{OTG-AUTHN-005}
\end{itemize}

\subsection*{Tool 6}
\begin{itemize}
	\item \textbf{Used by}\\ Name Surname
	\item \textbf{Used for}\\ Sample Text - description of the tool
	\item \textbf{Useful in}\\ \vulnref{OTG-AUTHN-004} and \vulnref{OTG-AUTHN-005}
\end{itemize}

\subsection*{Tool 7}
\begin{itemize}
	\item \textbf{Used by}\\ Name Surname
	\item \textbf{Used for}\\ Sample Text - description of the tool
	\item \textbf{Useful in}\\ \vulnref{OTG-AUTHN-004} and \vulnref{OTG-AUTHN-005}
\end{itemize}

\subsection*{Tool 8}
\begin{itemize}
	\item \textbf{Used by}\\ Name Surname
	\item \textbf{Used for}\\ Sample Text - description of the tool
	\item \textbf{Useful in}\\ \vulnref{OTG-AUTHN-004} and \vulnref{OTG-AUTHN-005}
\end{itemize}

\clearpage

TODO:\newline
describe identified vulnerabilities mentioning the following for each one
\begin{itemize}
	\item Observation: which part of the application is vulnerable and why?
	\item Discovery: how was this vulnerability discovered? Which tools were used and which steps were performed?
	\item Likelihood: what is the likelihood that this vulnerability is exploited? Which assumptions must hold and which skills must an attacker have?
	\item Impact: what is the potential impact of an exploit of this vulnerability? What could happen?
	\item Recommendation: how exactly would you fix this problem?
	\item Instantiate the CVSS v3.0 template for this particular vulnerability.
	\item Comparison with your own application: how and why is your app better/worse
\end{itemize}

NOTR: Each vulnerability should be described on a new page!\newline

\clearpage
Sample tables
\vultable{\bs}{%
	Observation
}{%
Discovery
}{%
Likelihood
}{%
Implications
}{%
Recommendation
}{%
\naScore
%\cvssBaseScorePretty{N}{H}{N}{R}{U}{H}{H}{N}
}
\vultable{\gnb}{%
	Observation
}{%
Discovery
}{%
Likelihood
}{%
Implications
}{%
Recommendation
}{%
\naScore
%\cvssBaseScorePretty{N}{H}{N}{R}{U}{H}{H}{N}
}
\clearpage

\section{Configuration and Deploy Management Testing}
\simpleVulntitle{OTG-CONFIG-003}{Test File Extensions Handling for Sensitive Information}
\vultable{\bs}{%
	We discovered that it is possible to access some files and directories which should not be accessible to the user/attacker. Specifically, we were able to get a hold of the private key used by the server for SSL encryption, the MySQL dump file and other data that should not be accessible from the network and, even less, to attackers.
}{%
	We managed to directly download several files containing sensitive information directly, as well as access some unprotected content:
	\begin{itemize}
		\item \texttt{http://[HOST]/app/HTTPS/hostonly.key}
		\item \texttt{http://[HOST]/adminer}
		\item \texttt{http://[HOST]/dump.sql}
		\item \texttt{http://[HOST]/Smart-Card-Simulator/src/scs/Main.java}, as well as all other Java sources.
	\end{itemize}
	Also many other README and temporary files are directly accessible.\newline
}{%
	Since directory listing is disabled inside the /api and /app subfolders of the application, most of the mentioned files can be accessed only via brute-force attacks, and even then it would prove very difficult to guess the name of the directory and filename correctly. During white-box testing, this was not an issue, but in reality, finding out this weakness would require many attempts and much time.\newline
	Other names of files, however, contained in the root folder (directory listing here is not disabled), like dump.sql can be easily guessed.
}{%
	Although an attacker cannot directly compromise the integrity and availability of the server, it is possible to access some very important content of the application. In particular, getting a hold of \texttt{hostonly.key} allows to start MITM attacks on any victim, since the file contains the private key of the server used for SSL encryption.
}{%
	Relocate the .htaccess files (or the root of the web application) and place more strict rules for file/directory access, since these are too loose.
}{%
	\cvssBaseScorePretty{N}{H}{N}{N}{U}{H}{N}{N}
}
\vultable{\gnb}{%
	TODO: Observation
}{%
	TODO: Discovery
}{%
	TODO: Likelihood
}{%
	TODO: Implications
}{%
	TODO: Recommendation
}{%
	\naScore
	%\cvssBaseScorePretty{N}{H}{N}{R}{U}{H}{H}{N}
}

\vulntitle{OTG-CONFIG-006}{Test HTTP Methods}
\vulntitle{OTG-CONFIG-007}{Test HTTP Strict Transport Security}
\vultable{\bs}{%
	Although HTTPS is enabled on port 443, access to the page without using HTTPS is still possible, by simply accessing the server on port 80.
}{%
	We discovered this by simply trying to access the application on port 80: this way we had normal access to the application, although without any transport security.
}{%
	If any user accesses port 80 instead of port 443 (e.g. by mistake), his whole connection to the server will be unencrypted, which makes this vulnerability highly likely. Additionally an attacker might be able to redirect a victim to port 80, in order to intercept all the traffic. This requires some social engineering but is still feasible.\newline
	Once having exploited this vulnerability, the likelihood of any attack which assumes lack of transport security becomes very high.
}{%
	The implications of lack of an encrypted connection were already analysed in phase 2. To review the implications briefly: all the data is sent in clear text from the client to the server and vice-versa, allowing man in the middle attacks. Confidentiality, integrity and authenticity are not fullfilled, allowing to hijack a session, view user data (including passwords) and modify it at will.
}{%
	Deactivate the default port 80, thus allowing only connections on port 443.
}{%
	\cvssBaseScorePretty{N}{L}{N}{R}{U}{H}{H}{N}
}
\vultable{\gnb}{%
	This vulnerability was fixed in phase 3 and the application is only allowing communication between the client and the server via HTTPS.
}{%
	We discovered this by simply testing to access both ports 80 and 443 on the webserver, which only allowed us to connect to the latter. A connection on port 80 was refused by the server. We also enforced this theory by checking it with nmap, getting the same result.
}{%
	\na
}{%
	\na
}{%
	\na
}{%
	\secure
}

\vulntitle{OTG-CONFIG-008}{Test RIA cross domain policy}

\clearpage
\section{Identity Management Testing}
\simpleVulntitle{OTG-IDENT-001}{Test Role Definitions}
\vulntitle{OTG-IDENT-002}{Test User Registration Process}
\vulntitle{OTG-IDENT-003}{Test Account Provisioning Process}
\vulntitle{OTG-IDENT-004}{Testing for Account Enumeration and Guessable User Account}
\vulntitle{OTG-IDENT-005}{Testing for Weak or unenforced username policy}

\clearpage
\section{Authentication Testing}
\simpleVulntitle{OTG-AUTHN-001}{Testing for Credentials Transported over an Encrypted Channel}
\vulntitle{OTG-AUTHN-002}{Testing for default credentials}
\vulntitle{OTG-AUTHN-003}{Testing for Weak lock out mechanism}
\vulntitle{OTG-AUTHN-004}{Testing for bypassing authentication schema}
\vulntitle{OTG-AUTHN-005}{Test remember password functionality}
\vulntitle{OTG-AUTHN-006}{Testing for Browser cache weakness}
\vulntitle{OTG-AUTHN-007}{Testing for Weak password policy}
\vulntitle{OTG-AUTHN-008}{Testing for Weak security question/answer}
\vulntitle{OTG-AUTHN-009}{Testing for weak password change or reset functionalities}
\vulntitle{OTG-AUTHN-010}{Testing for Weaker authentication in alternative channel}

\clearpage
\section{Authorization Testing}
\simpleVulntitle{OTG-AUTHZ-001}{Testing Directory traversal/file include}
\vulntitle{OTG-AUTHZ-002}{Testing for bypassing authorization schema}
\vulntitle{OTG-AUTHZ-003}{Testing for Privilege Escalation}
\vulntitle{OTG-AUTHZ-004}{Testing for Insecure Direct Object References}

\clearpage
\section{Session Management Testing}
\simpleVulntitle{OTG-SESS-001}{Testing for Bypassing Session Management Schema}
\vulntitle{OTG-SESS-002}{Testing for Cookies attributes}
\vulntitle{OTG-SESS-003}{Testing for Session Fixation}
\vulntitle{OTG-SESS-004}{Testing for Exposed Session Variables}
\vulntitle{OTG-SESS-005}{Testing for Cross Site Request Forgery}
\vulntitle{OTG-SESS-006}{Testing for logout functionality}
\vulntitle{OTG-SESS-007}{Test Session Timeout}
\vulntitle{OTG-SESS-008}{Testing for Session puzzling}

\clearpage
\section{Data Validation Testing}
\simpleVulntitle{OTG-INPVAL-001}{Testing for Reflected Cross Site Scripting}
\simpleVulntitle{OTG-INPVAL-002}{Testing for Stored Cross Site Scripting}
\vulntitle{OTG-INPVAL-003}{Testing for HTTP Verb Tampering}
\vulntitle{OTG-INPVAL-004}{Testing for HTTP Parameter pollution}
\vulntitle{OTG-INPVAL-005}{Testing for SQL Injection}
\vulntitle{OTG-INPVAL-006}{Testing for LDAP Injection}
\vulntitle{OTG-INPVAL-007}{Testing for ORM Injection}
\vulntitle{OTG-INPVAL-008}{Testing for XML Injection}
\vulntitle{OTG-INPVAL-009}{Testing for SSI Injection}
\vulntitle{OTG-INPVAL-010}{Testing for XPath Injection}
\vulntitle{OTG-INPVAL-011}{IMAP/SMTP Injection}
\vulntitle{OTG-INPVAL-012}{Testing for Code Injection}
\vulntitle{OTG-INPVAL-012-1}{Testing for Local File Inclusion}
\vulntitle{OTG-INPVAL-012-2}{Testing for Remote File Inclusion}
\vulntitle{OTG-INPVAL-013}{Testing for Command Injection}
\vultable{\bs}{%
	We observed that it is possible to perform command injection attacks, by exploiting the batch transaction functionality, 
	as no input validation upon the values inserted by the user in the transaction description field is performed. \newline
	Regardless of the type of used banking method (either SCS or pre-generated TANs), as long as a user inserts a valid TAN when performing a transaction, it will also be possible to inject commands to the target system through the description field.
}{%
	This was discovered while analyzing the \texttt{uploadFile} callback function in the index.php file (lines 95-154).\newline
	Since the application doesn't sanitize user input (neither on client nor on server side), the \texttt{exec} function called from the server to execute the C parser can be exploited to execute arbitrary commands, by simply inserting them into the description field. These commands will simply be executed after the parser. Here is an example:\newline
	\texttt{test"; ls -l; exit 1 \#} \newline
	\textbf{Note:} since the output of the \texttt{exec} operation is only visible on client side if the return code was different than 0 (hence on transaction failure), we just need to force a return code (as shown in the example above).
}{%
	This kind of attack may require several brute-force attempts and has therefore medium likelihood, but once found out, the vulnerability is easy to exploit.
}{%
	The implications of this attack are high, since the attacker is able to execute arbitrary commands on the target system. It is important to stress though, that it is not possible to modify any file in the target web directory, due to insufficient privileges.
}{%
	It is highly recommended to sanitize the description value inserted by the user, or to make the user write the descriptions directly inside the batch file, avoiding thus command injections.
}{%
	\cvssBaseScorePretty{N}{L}{L}{N}{U}{L}{N}{H}
}
\vultable{\gnb}{%
	Observation
}{%
Discovery
}{%
Likelihood
}{%
Implications
}{%
Recommendation
}{%
\naScore
%\cvssBaseScorePretty{N}{H}{N}{R}{U}{H}{H}{N}
}

\vulntitle{OTG-INPVAL-014}{Testing for Buffer overflow}
\vulntitle{OTG-INPVAL-014-2}{Testing for Stack overflow}
\vulntitle{OTG-INPVAL-014-3}{Testing for Heap overflow}
\vulntitle{OTG-INPVAL-014-4}{Testing for Format string}
\vulntitle{OTG-INPVAL-015}{Testing for incubated vulnerabilities}
\vulntitle{OTG-INPVAL-016}{Testing for HTTP Splitting/Smuggling}

\clearpage
\section{Error Handling}
\simpleVulntitle{OTG-ERR-001}{Analysis of Error Codes}
\vulntitle{OTG-ERR-002}{Analysis of Stack Traces}

\clearpage
\section{Cryptography}
\simpleVulntitle{OTG-CRYPST-001}{Testing for Weak SSL/TSL Ciphers, Insufficient Transport Layer Protection}
\vulntitle{OTG-CRYPST-002}{Testing for Padding Oracle}
\vulntitle{OTG-CRYPST-003}{Testing for Sensitive information sent via unencrypted channels}

\clearpage
\section{Business Logic Testing}
\simpleVulntitle{OTG-BUSLOGIC-001}{Test Business Logic Data Validation}
\vulntitle{OTG-BUSLOGIC-002}{Test Ability to Forge Requests}
\vulntitle{OTG-BUSLOGIC-003}{Test Integrity Checks}
\vulntitle{OTG-BUSLOGIC-004}{Test for Process Timing}
\vulntitle{OTG-BUSLOGIC-005}{Test Number of Times a Function Can be Used Limits}
\vulntitle{OTG-BUSLOGIC-006}{Testing for the Circumvention of Work Flows}
\vulntitle{OTG-BUSLOGIC-007}{Test Defenses Against Application Mis-use}
\vulntitle{OTG-BUSLOGIC-008}{Test Upload of Unexpected File Types}
\vulntitle{OTG-BUSLOGIC-009}{Test Upload of Malicious Files}

\clearpage
\section{Client Side Testing}
\simpleVulntitle{OTG-CLIENT-001}{Testing for DOM based Cross Site Scripting}
\vulntitle{OTG-CLIENT-002}{Testing for JavaScript Execution}
\vulntitle{OTG-CLIENT-003}{Testing for HTML Injection}
\vulntitle{OTG-CLIENT-004}{Testing for Client Side URL Redirect}
\vulntitle{OTG-CLIENT-005}{Testing for CSS Injection}
\vulntitle{OTG-CLIENT-006}{Testing for Client Side Resource Manipulation}
\vulntitle{OTG-CLIENT-007}{Test Cross Origin Resource Sharing}
\vulntitle{OTG-CLIENT-008}{Testing for Cross Site Flashing}
\vulntitle{OTG-CLIENT-009}{Testing for Clickjacking}
\vulntitle{OTG-CLIENT-010}{Testing WebSockets}
\vulntitle{OTG-CLIENT-011}{Test Web Messaging}
\vulntitle{OTG-CLIENT-012}{Test Local Storage}
