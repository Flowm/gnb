\chapter{Bugs}\label{chapter:bugs}
\bugtable{1}{BugName}{%
	Location
}{%
Referenced Vulnerabilities
}{%
Discription
}{%
How to trigger
}

see \bugref{1}.

1) There is not reference whatsoever to how a batch file should be like. Neither the required file extension nor the format of the file contents, are mentioned in any way. It is only possible to access this information by browsing through the directories of the web application.
24	Florian	Manual	BUG	done	/tran.php (batch)	Sending a Transaction by batchfile	Batch fromat not documented																					

\hrulefill \newline
5) When inserting an invalid TAN, the server sends an error message in which the expected valid TAN is sent back to client in clear text.

\hrulefill \newline
7) TAN codes appear to be randomly generated. However, when generating TAN codes, the user id and the number of the code are prepended to the string. So, if a user ID was of length 5 and the number of a code of length 2, the actual length of the code, which was randomly generated, would be 15 -5 -2 = 8. An attcker could register an arbitrary number of users to make the user ID counter go up inside the database (auto-increment is used). Once the counter is reasonably high, the TAN code complexity would reduce drastically, allowing to brute-force them easily.

\hrulefill \newline
8) The size limit of a batch transaction file is of 500 bytes. This is ridiculous for multiple transaction purposes, since it allows to actually insert only up to 3 transactions at once.
14	Alex	Manual	BUG	done	/tran.php	Can not upload batch files with more than 500 Byte size		

\hrulefill \newline
18	Florian	Manual	BUG	todo	/employee\_home.php	Clicking on home as employee	Home button redirecting to /employee\_home without php, which results in 404																					