\chapter{Fixes}\label{chapter:fixes}

For each vulnerability discovered in Phase 4 by your team or the team that tested your
application, you should summarize the fix (security measure) that you implemented:
\begin{itemize}
	\item Path(s) of file(s) which were modified for the fix
	\item Line number(s) which were modified in each file 
	\item NOTE: You can use a diff utility but you should only report the changes relevant for one particular fix per vulnerability (not just a diff of all changes to phase 3).
	\item Textual description of your countermeasures and why it fixes the problem (i.e. the effect that the modifications have against attack).
\end{itemize}

Sample table
\fixtable{%
	%Description of the vulnerability
}{%
	%Cvss score
}{%
	%Modified files (paths)
}{%
	%Modified line numbers
}{%
	%Countermeasures (description)
}

\section{Configuration and Deploy Management Testing}
\simpleVulntitle{OTG-CONFIG-001}{Test File Extensions Handling for Sensitive Information}

\clearpage
\section{Identity Management Testing}
\simpleVulntitle{OTG-IDENT-002}{Test User Registration Process}
\vulntitle{OTG-IDENT-003}{Test Account Provisioning Process}

\clearpage
\section{Error Handling}
\simpleVulntitle{OTG-ERR-001}{Analysis of Error Codes}

\clearpage
\section{Cryptography}
\simpleVulntitle{OTG-CRYPST-001}{Testing for Weak SSL/TSL Ciphers, Insufficient Transport Layer Protection}

\clearpage
\section{Business Logic Testing}
\simpleVulntitle{OTG-BUSLOG-001}{Test Business Logic Data Validation}
\vulntitle{OTG-BUSLOG-006}{Testing for the Circumvention of Work Flows}
\vulntitle{OTG-BUSLOG-007}{Test Defenses Against Application Mis-use}