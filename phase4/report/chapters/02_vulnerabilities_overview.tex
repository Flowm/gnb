\chapter{Vulnerabilities Overview}\label{chapter:vulnerabilities_overview}
This section will discuss the main vulnerabilities found in the \bs application by Team 12, it will categorize, describe and score each vulnerability according to CVSS 3.0.\newline

%THEIR SECTION
\section{\bs}
\subsection{Directory listing and file extensions handling} \label{over:vuln_1}
\begin{itemize}
	\item CVSS Score: \textit{7.5}
	\item Likelihood: \textit{low}
	\item Impact: \textit{high}
	\item Risk: \textit{high}
	\item Reference: OWASP \vulnref{OTG-CONFIG-003}
\end{itemize}
Some directories can be accessed by the browser directly, as well as some sensitive files, like the database dump file and the SSL private key. This is due to the loose Apache configuration policies.

\subsection{Guessable user account} \label{over:vuln_2}
\begin{itemize}
	\item CVSS Score: \textit{5.3}
	\item Likelihood: \textit{low}
	\item Impact: \textit{low}
	\item Risk: \textit{low}
	\item Reference: OWASP \vulnref{OTG-IDENT-004}
\end{itemize}
The server returns different error codes after a login attempt, depending on whether the inserted username or the password were incorrect. This makes it easier to guess if a user account exists on the database or not.

\subsection{Default admin credentials} \label{over:vuln_3}
\begin{itemize}
	\item CVSS Score: \textit{5.4}
	\item Likelihood: \textit{high}
	\item Impact: \textit{medium}
	\item Risk: \textit{medium}
	\item Reference: OWASP \vulnref{OTG-AUTHN-002}
\end{itemize}
The default admin credentials can be easily brute-forced by a malicious attacker, granting them employee rights inside the \bs{} application.

\subsection{Insecure cookies} \label{over:vuln_4}
\begin{itemize}
	\item CVSS Score: \textit{5.4}
	\item Likelihood: \textit{medium}
	\item Impact: \textit{medium}
	\item Risk: \textit{high}
	\item Reference: OWASP \vulnref{OTG-SESS-002}
\end{itemize}
The application uses insecure cookies, which can be accessed by a malicious attacker. Since the application mainly relies on the session ID cookie for every operation, it is possible to hijack an existing session and access the client/employees accounts.

\subsection{Cross Site Request Forgery} \label{over:vuln_5}
\begin{itemize}
	\item CVSS Score: \textit{4.2}
	\item Likelihood: \textit{low}
	\item Impact: \textit{low}
	\item Risk: \textit{low}
	\item Reference: OWASP \vulnref{OTG-SESS-005}
\end{itemize}
By setting a custom value inside the XSRF cookie and the value inside the GET request, it is possible to perform XSRF attacks. This is due to an incorrect comparison between the cookie and the GET parameter passed to the server.

\subsection{Logout functionality} \label{over:vuln_6}
\begin{itemize}
	\item CVSS Score: \textit{4.2}
	\item Likelihood: \textit{low}
	\item Impact: \textit{medium}
	\item Risk: \textit{medium}
	\item Reference: OWASP \vulnref{OTG-SESS-006}
\end{itemize}
The adopted logout mechanism invalidates the cookie on the client side, but does not correctly invalidate the session on server side. If an attacker gained access to a session cookie before this was deleted, this could lead to a session hijacking scenario.

\subsection{Command injection} \label{over:vuln_7}
\begin{itemize}
	\item CVSS Score: \textit{9.6}
	\item Likelihood: \textit{high}
	\item Impact: \textit{high}
	\item Risk: \textit{high}
	\item Reference: OWASP \vulnref{OTG-INPVAL-013}
\end{itemize}
It is possible to exploit the batch transaction functionality by injecting arbitrary bash commands directly into the description field, which will then be executed by the application.
This also allows to view all the source code of the application, including some sensitive informations (e.g. db password).

\subsection{Heap overflow} \label{over:vuln_8}
\begin{itemize}
	\item CVSS Score: \textit{3.3}
	\item Likelihood: \textit{low}
	\item Impact: \textit{low}
	\item Risk: \textit{low}
	\item Reference: OWASP \vulnref{OTG-INPVAL-014-3}
\end{itemize}
This vulnerability affects the C parser, although it cannot be directly exploited via the web interface. Due to incorrect \texttt{strncpy} input, it is possible to produce heap overflows with the correct parameters, overriding thus some memory areas.

\subsection{Insufficient transport layer protection} \label{over:vuln_9}
\begin{itemize}
	\item CVSS Score: \textit{6.8}
	\item Likelihood: \textit{high}
	\item Impact: \textit{high}
	\item Risk: \textit{high}
	\item Reference: OWASP \vulnref{OTG-CRYPST-001}
\end{itemize}
Due to a missing openssl update, the application is subject to the Heartbleed vulnerability. Also, because SSLv3 is not disabled, the application is subject to the Poodle bug.

%OUR SECTION
\section{\gnb}
\subsection{Insufficient transport layer protection} \label{over:vuln_10}
\begin{itemize}
	\item CVSS Score: \textit{3.1}
	\item Likelihood: \textit{high}
	\item Impact: \textit{high}
	\item Risk: \textit{high}
	\item Reference: OWASP \vulnref{OTG-CRYPST-001}
\end{itemize}
Because SSLv3 is not disabled, the application is subject to the Poodle bug.


\subsection{Defenses Against Application Mis-use} \label{over:vuln_11}
\begin{itemize}
	\item CVSS Score: \textit{7.1}
	\item Likelihood: \textit{high}
	\item Impact: \textit{low}
	\item Risk: \textit{low}
	\item Reference: OWASP \vulnref{OTG-BUSLOGIC-007}
\end{itemize}
No mechanisms to prevent against application mis-use are in place except the lockout-functionality (see \vulnref{OTG-AUTHN-003}). No critical functionalities are disabled and no logs are kept.

\subsection{File Extensions Handling for Sensitive Information} \label{over:vuln_12}
\begin{itemize}
	\item CVSS Score: \textit{3.1}
	\item Likelihood: \textit{low}
	\item Impact: \textit{low}
	\item Risk: \textit{low}
	\item Reference: OWASP \vulnref{OTG-CONFIG-003}
\end{itemize}
The Apache server is configured globally for the whole application, disabling directory listing but allowing direct access to all files inside the web folder. This folder mainly contains PHP, Javascript, HTML, CSS and media files (images). Among these, no sensitive information can be leaked. We found, however, that the upload folder contents are potentially accessible. So, if an attacker could brute-force the name of an uploaded file (which is entirely random) before it gets deleted by the server, he could read the contents.	

\section{Comparison}
The major threats on the \bs{} application come from its vulnerability to Command Injection and Heartbleed both of which have a high CVSS v3 score.

While the \gnb{} applications major vulnerability was its susceptibility to the SSL Poodle attack.

While both application are not production ready due to the vulnerabilities available,  the \gnb{} application has shown more promise and is far more resilient to attacks than the \bs{} application.

