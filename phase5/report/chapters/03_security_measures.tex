\chapter{Security Measures}\label{chapter:security_measures}
\todoin{Short intro text here}
\todoin{Description: Enumerate the security features your application uses and which attacks each feature defends against (specify which features you implemented and which you borrowed/use from other libraries)}

%Sample table
\securityMeasure{%
	%description
}{
	%implementation
}{
	%safe against
}

\clearpage
\subsection{HTTPS with HSTS}
\securityMeasure{%
	The site is only reachable over HTTPS and employs HSTS (HTTP Strict
	Transport Security) to protect against a downgrade of future connection
	attempts to HTTP.
}{%
	The apache2 server was configured in a way that only allows connections via
	HTTPS and always appends the \texttt{Strict-Transport-Security} header to
	replys to enable HSTS in the clients browser.
}{%
	\begin{itemize}
		\item Capturing of the communication between the client and the server
			by an third party.
		\item Modification of the communication between the client and the
			server by a man-in-the-middle attacker.
		\item Downgrade of a new connection to the server to HTTP by a
			man-in-the-middle attacker (after an initial connection to the correct
			site)
	\end{itemize}
}


\clearpage
\subsection{SSL secure ciphers}\label{sec:ssl_ciphers}
\securityMeasure{%
	The SSL ciphers offered by the webserver for HTTPS connections are
	configured to confirm the latest standards.
}{%
	The apache2 server was configured to only offer ciphers that are
	known secure.
	As visible in \autoref{figure:ciphers1} and \autoref{figure:ciphers2},
	current SSL testing tools give the site a perfect score.
}{%
	Save against SSL vulnerabilities. E.g.\ Heartbleed, POODLE, FREAK.
	Prevents eavesdropping and man-in-the-middle attacks caused by weak ciphers.
}
\begin{figure}[h!tbp]
	\centering
	\includegraphics[width=\textwidth]{figures/ciphers1.png}
	\caption{Available SSL Ciphers}
	\label{figure:ciphers1}
\end{figure}
\begin{figure}[h!tbp]
	\centering
	\includegraphics[width=\textwidth]{figures/ciphers2.png}
	\caption{No known SSL Vulnerabilities}
	\label{figure:ciphers2}
\end{figure}


\clearpage
\subsection{CSRF tokens}
\securityMeasure{
	The application automatically creates anti-CSRF cryptographic nonces (tokens) on all pages with forms that could be exploited to perform operations using the profile of a user. These tokens are validated once the form is submitted by the user, ensuring that the application can't be forced to perform actions via cross-site requests.
}{
	This feature was implemented manually, by creating a 256-bit random token (obtained via the PHP builtin \texttt{openssl\_random\_pseudo\_bytes} function) and encoding it base64. This token is then saved as a session variable and sent to the user as a hidden field inside a form, as can be seen in the figure below. Once the form is submitted by the user, the the token contained inside the form is compared against the one already existing on the server.
}{
	This particular security feature protects against XSRF attacks, as described in OTG-SESS-005.
}
\begin{figure}[h!tbp]
	\centering
	\includegraphics[width=\textwidth]{figures/csrf_token.png}
	\caption{CSRF token generated in the transaction page}
	\label{figure:csrf}
\end{figure}


\clearpage
\subsection{Password strength}
%Sample table
\securityMeasure{%
	We enforce strong passwords by forcing the users to respect the following criteria:
	\begin{itemize}
		\item length between 8 and 20 characters;
		\item at least 1 numeric character (0-9);
		\item at least 1 upper/lower case letter (a-zA-Z);
		\item any special character inside the password is allowed.
	\end{itemize}
	Any password not matching these criteria is rejected by the server.
}{
	The password enforcing mechanism was manually implemented both on client side (see \texttt{project/js/registration.js}) and on server side (see \texttt{project/registration/registration\_request.php}). The user is compelled to follow the mentioned criteria when choosing a password during registration. This also holds true when resetting a password later on.
}{
	Strong password prevent attackers from guessing them, as described in OTG-AUTHN-007.
}


\clearpage
\subsection{Secure passwords}
\securityMeasure{%
	User passwords are stored inside the database after a successful registration. Passwords are never stored in plaintext, but are hashed using a random salt, unique for every user, together with an additional static salt, known only to the backend server. 	
}{
	The algorithm used for hashing these values together is the SHA-512. A salt is generated for every user via the builtin PHP \texttt{openssl\_random\_pseudo\_bytes} function. The salt and the hashed password are both stored on the database (see \autoref{section:db}). No additional libraries were used to implement this feature.
}{
	Secure passwords protect the users in case the database is compromised in any way. An attacker is not able to determine which users are using the same password and cannot retrieve the original password, starting from the hashed one.
}


\clearpage
\subsection{Password reset with two-factor authentication}
\securityMeasure{%
	The user has the possibility to reset his password in case he does not remember it. A password reset requires the users mail address and access to the used mail account. Additionally the user has to enter his PIN to complete the process of resetting his password. This prevents attackers that already have access to the users mail address from resetting the password.
}{
	This has been implemented using the columns \texttt{pw\_reset\_hash}, \texttt{pw\_reset\_hash\_timestamp} and \texttt{pin} in the table \texttt{user}. When a password reset is requested a mail is sent to the user with a randomly generated token which is also stored in the database with the current timestamp. The user then has to use this token before it is invalidated by clicking on the link in the email, and providing a new password and the current PIN in the form.
}{
	This security feature protects users from attackers that already have access to their mail account, physical access to their device(s) or started the password reset process through social engineering.
}


\clearpage
\subsection{Lockout mechanism after failed password entries}
\securityMeasure{%
	In order to use our website users are required to log in. For a successful login the right username (which is the users mail address) and the correct password are required. To prevent brute force attacks given an already known mail address a lockout mechanism has been implemented. This mechanism blocks the user account if the wrong password has been entered 5 times in a row. The user has to be unblocked by an employee afterwards.

The same functionality was implemented to prevent brute force attacks on TANs for transactions.
}{
	This feature was implemented by adding the table \texttt{failed\_attempts} to the database (see \autoref{section:db}). This table stores all failed attempts (e.g. login or an invalid TAN) with the timestamp of the attempt and the ID of the user that executed the action. Every time an invalid password or TAN is observed a failed attempt is added the database and the sum of all failed attempts is checked. Once a valid password or TAN is entered all failed attempts are reset.
}{
	This feature protects against Application Mis-Use (OTG-BUSLOGIC-007) and brute force attacks on the login functionality (OTG-AUTHN-003).
}


\clearpage
\subsection{Automatic logout after inactivity}
\securityMeasure{%
	The application automatically logs out users who have been inactive for 30
	minutes.
}{%
	This was implemented on two different levels. The
	\texttt{session.gc\_maxlifetime} setting in the php configuration allows
	sessions to be removed after 1440 seconds. Additionally the user is logged
	out by a check in the php code if it latest activity lies more than 30
	minutes in the past.
}{%
	Prevents stealing of the user session from the clients computer after the
	timeout.
}


\clearpage
\subsection{CAPTCHAs}
\securityMeasure{%
	During the registration process users are required to enter a CAPTCHA code in order to prevent malicious attackers from automating registrations. This CAPTCHA is entirely random and requires the user to read an image and input the alphanumeric code manually.
}{
	For the creation of the CAPTCHA we resort to the \texttt{secureimage} library (see \ref{chapter:application_architecture} for further info).
}{
	This security feature protects the application from attackers who try to automate registrations (see OTG-IDENT-002), which could lead to a database saturation or even to a DOS.
}


\clearpage
\subsection{Balance calculated in database (Realtime concurrent transactions)}
\securityMeasure{%
	As described in \autoref{section:db} the balance for each given account is always calculated in realtime in the database. The \texttt{VIEW} \texttt{account\_overview} uses all existing transactions and determines its value for the given account using the transaction status, the destination account and source account. This not only ensures an always consistent database because it is not possible to remove the balance for one account and "forget" to update the balance for the other account, but also enables realtime concurrent transactions, because no locking is necessary when new transactions are added.
}{
	This feature was implemented using the SQL-\texttt{VIEW} \texttt{account\_overview} which combines the tables \texttt{account} and \texttt{transaction} to calculate the current balance for a  given account. The implementation of this \texttt{VIEW} is shown in \autoref{figure:accountoverviewfigure} and explained in \autoref{section:db}.
}{
	This security feature protects against flaws in the business logic data validation (OTG-BUSLOGIC-001) e.g. if actions are not completed successfully without taking necessary rollback steps. This also prevents exploiting a possible Circumvention of Work Flow (OTG-BUSLOGIC-006) as one transaction is an atomar operation. Summed up this feature ensures that the database is always consistent and no money can get lost due to incompletely handled transactions.
}


\clearpage
\subsection{Password protected PDFs for TAN lists}
\securityMeasure{%
	The TAN list sent out to new users upon approval is encrypted with a
	password which gets sent to the user over a separate channel.
}{
	We encrypt the TAN list PDF with a password derived from the user pin. The
	user gets this password displayed in the webinterface when he successfully
	logs in.
}{
	Interception of the TAN list as it's sent over an unencrypted channel
	(email).
}


\clearpage
\subsection{Time-based TANs generated via SCS}

\clearpage
\subsection{Secure cookies}
\securityMeasure{%
	%description
}{
	%implementation
}{
	%safe against
}

\clearpage
\subsection{Input sanitization}
prevents XSS, command injection and sql injection
\securityMeasure{%
	%description
}{
	%implementation
}{
	%safe against
}

\clearpage
\subsection{Prepared statements}
\securityMeasure{%
	%description
}{
	%implementation
}{
	%safe against
}

\clearpage
\subsection{Clickjacking prevention}
X-Frame-Options
\securityMeasure{%
	%description
}{
	%implementation
}{
	%safe against
}